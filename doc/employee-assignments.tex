\documentclass{article}
\usepackage{amsmath,amssymb}
\usepackage{comment}

\setlength\parindent{0pt}

\begin{document}

\tableofcontents

\section{Problem Statement}

Generalize over the constant $$T$$
with constraints:
\[
O + E + B + M \leq T
\]
\[
O, E, B, M \geq 0
\]

to maximize:
\[
f(O, E, B, M) = \left( E^{0.34} + O^{0.2} \right) \left( 1 + \frac{M}{1.2 \left( O + E + B + M \right)} \right)
\]

\section{Analysis over O and E}

\subsection{Simplification}

First, observe that B cannot have a positive effect on the objective function, so we assign $B = 0$:

\[
f(O, E, M) = \left( E^{0.34} + O^{0.2} \right) \left( 1 + \frac{M}{1.2 \left( O + E + M \right)} \right)
\]

Then observe that increasing $M$ without violating $O + E + B + M \leq T$ can only have a positive effect on the objective function, so we can assume $T = O + E + M$. Use this to simplify the objective function by substituting $M = T - O - E$:

\[
f(O, E) = \left( E^{0.34} + O^{0.2} \right) \left( 1 + \frac{T - O - E}{1.2 T} \right)
\]

\subsection{Optimization}

To optimize the objective function, we take the partial derivatives with respect to $O$ and $E$.

\subsubsection{Find partial derivatives}

Useg the product rule $\frac{\partial}{\partial O}(uv) = \frac{\partial u}{\partial O}v \cdot u\frac{\partial v}{\partial O}$. In

\begin{equation}\label{eq:fuv}
  f(O, E) = u \cdot v
\end{equation}
let
\begin{equation}\label{eq:u}
  u = E^{0.34} + O^{0.2}
\end{equation}
and
\begin{equation}\label{eq:v}
  v = 1 + \frac{T - O - E}{1.2 T}
\end{equation}

First, $\frac{\partial f}{\partial O}$. The derivatives are:
\begin{equation}\label{eq:dudo}
  \frac{\partial u}{\partial O} = 0.2O^{-0.8}
\end{equation}
and
\begin{equation}\label{eq:dvdo}
  \frac{\partial v}{\partial O} = -\frac{1}{1.2T}
\end{equation}

From \eqref{eq:fuv}, \eqref{eq:u}, and \eqref{eq:v}, we have:
\begin{equation}
  \begin{split}
    f(O, E) & = u \cdot v \\
    \span \text{Use sloppy notation $\frac{\partial f}{\partial E}$ = $\frac{\partial}{\partial E}f(O,E)$ for brevity} \\
    \frac{\partial f}{\partial O} & = \frac{\partial u}{\partial O}v \cdot u + u\frac{\partial v}{\partial O}
  \end{split}
\end{equation}

Substitute \eqref{eq:dudo}, \eqref{eq:v}, \eqref{eq:u}, and \eqref{eq:dvdo} into the above equation:
\begin{equation}\label{eq:dfdo}
  \begin{split}
  \frac{\partial f}{\partial O} & = 0.2O^{-0.8}\left(1 + \frac{T - O - E}{1.2T}\right) + \left(E^{0.34} + O^{0.2}\right)\frac{-1}{1.2T} \\
  & = 0.2O^{-0.8}\left(1 + \frac{T - O - E}{1.2T}\right) - \frac{1}{1.2T}\left(E^{0.34} + O^{0.2}\right)
  \end{split}
\end{equation}

Repeat this process for $\frac{\partial f}{\partial E}$. The derivatives are:
\begin{equation}\label{eq:dude}
  \frac{\partial u}{\partial E} = 0.34E^{-0.66}
\end{equation}
and
\begin{equation}\label{eq:dvde}
  \frac{\partial v}{\partial E} = -\frac{1}{1.2T}
\end{equation}

From \eqref{eq:fuv}, \eqref{eq:u}, and \eqref{eq:v}, we have:
\begin{equation}
  \begin{split}
    f(O, E) & = u \cdot v \\
    \frac{\partial f}{\partial E} & = \frac{\partial u}{\partial E}v \cdot u + u\frac{\partial v}{\partial E}
  \end{split}
\end{equation}

Substitute \eqref{eq:dude}, \eqref{eq:v}, \eqref{eq:u}, and \eqref{eq:dvde} into the above equation:
\begin{equation}\label{eq:dfde}
  \begin{split}
  \frac{\partial f}{\partial E} & = 0.34E^{-0.66}\left(1 + \frac{T - O - E}{1.2T}\right) + \left(E^{0.34} + O^{0.2}\right)\frac{-1}{1.2T} \\
                                & = 0.34E^{-0.66}\left(1 + \frac{T - O - E}{1.2T}\right) - \frac{1}{1.2T}\left(E^{0.34} + O^{0.2}\right)
  \end{split}
\end{equation}

%\hrulefill

\subsubsection{Find critical points}

Set the partial derivatives \eqref{eq:dfdo} and \eqref{eq:dfde} to zero:
\begin{equation}\label{eq:dfdo0}
  0 = 0.2O^{-0.8}\left(1 + \frac{T - O - E}{1.2T}\right) - \frac{1}{1.2T}\left(E^{0.34} + O^{0.2}\right)
\end{equation}
and
\begin{equation}\label{eq:dfde0}
  0 = 0.34E^{-0.66}\left(1 + \frac{T - O - E}{1.2T}\right) - \frac{1}{1.2T}\left(E^{0.34} + O^{0.2}\right)
\end{equation}

Rearrange \eqref{eq:dfdo0} and \eqref{eq:dfde0} and dropping the common second summand:
\begin{equation}\label{eq:dfdo0r}
  0.2O^{-0.8}\left(1 + \frac{T - O - E}{1.2T}\right) = 0.34E^{-0.66}\left(1 + \frac{T - O - E}{1.2T}\right)
\end{equation}
divide both sides by $\left(1 + \frac{T - O - E}{1.2T}\right)$:
\begin{equation}\label{eq:dfdo0r2}
  0.2O^{-0.8} = 0.34E^{-0.66}
\end{equation}
divide both sides by $0.2$:
\begin{equation}\label{eq:dfdo0r3}
  O^{-0.8} = \frac{0.34E^{-0.66}}{0.2}
\end{equation}
Raise both sides to the power of $-1/0.8$, then simplifying:
\begin{equation}\label{eq:dfdo0r4}
  \begin{split}
    O & = \left(\frac{0.34E^{-0.66}}{0.2}\right)^{-1/0.8} \\
      & = \left(\frac{0.34}{0.2E^{0.66}}\right)^{-1/0.8} \\
      & = \left(\frac{0.34}{0.2E^{0.66}}\right)^{-1.25} \\
      & = \left(\frac{0.2E^{0.66}}{0.34}\right)^{1.25} \\
      & = \left(\frac{0.2}{0.34}\right)^{1.25}E^{0.825} \\
      \span \text{Let $k = \left(\frac{0.2}{0.34}\right)^{1.25}$, for simplicity} \\
      O & = kE^{0.825}
  \end{split}
\end{equation}

\subsubsection{Attempt to find closed form solution}

Substitute \eqref{eq:dfdo0r4} into \eqref{eq:dfde0}:
\begin{equation}\label{eq:dfde0r}
  \begin{split}
    0 & = 0.34E^{-0.66}\left(1 + \frac{T - O - E}{1.2T}\right) - \frac{1}{1.2T}\left(E^{0.34} + O^{0.2}\right) \\
      & = 0.34E^{-0.66}\left(1 + \frac{T - kE^{0.825} - E}{1.2T}\right) - \frac{1}{1.2T}\left(E^{0.34} + \left(kE^{0.825}\right)^{0.2}\right) \\
      & = 0.34E^{-0.66}\left(1 + \frac{T - kE^{0.825} - E}{1.2T}\right) - \frac{1}{1.2T}\left(E^{0.34} + k^{0.2}E^{0.165}\right) \\
      \span \text{Let $a = 0.34E^{-0.66}\left(1 + \frac{T - kE^{0.825} - E}{1.2T}\right)$,} \\
      \span \text{and $b=- \frac{1}{1.2T}\left(E^{0.34} + k^{0.2}E^{0.165}\right)$,} \\
      \span \text{to work with the smaller summands.} \\
      & = a + b \\
  \end{split}
\end{equation}

Expand and simplify $a$:
\begin{equation}\label{eq:simpa}
  \begin{split}
  a & = 0.34E^{-0.66} + \frac{0.34E^{-0.66}\left(T - kE^{0.825} - E\right)}{1.2T} \\
    & = 0.34E^{-0.66} + \frac{0.34}{1.2}E^{-0.66} - \frac{0.34k}{1.2T}E^{0.165} - \frac{0.34}{1.2T}E^{0.34} \\
    & = \frac{17}{50}E^{-0.66} + \frac{85}{3}E^{-0.66} - \frac{85k}{3T}E^{0.165} - \frac{85}{3T}E^{0.34} \\
    & = \frac{51}{150}E^{-0.66} + \frac{4250}{150}E^{-0.66} - \frac{85k}{3T}E^{0.165} - \frac{85}{3T}E^{0.34} \\
    & = \frac{4301}{150}E^{-0.66} - \frac{85k}{3T}E^{0.165} - \frac{85}{3T}E^{0.34} \\
  \end{split}
\end{equation}

Expand and simplify $b$:
\begin{equation}\label{eq:simpb}
  \begin{split}
  b & = - \frac{1}{1.2T}\left(E^{0.34} + k^{0.2}E^{0.165}\right) \\
    & = - \frac{1}{1.2T}E^{0.34} - \frac{k^{0.2}}{1.2T}E^{0.165} \\
    & = - \frac{5}{6T}E^{0.34} - \frac{5k^{0.2}}{6T}E^{0.165} \\
  \end{split}
\end{equation}

Substitute \eqref{eq:simpa} and \eqref{eq:simpb} into \eqref{eq:dfde0r}:
\begin{equation}\label{eq:dfde0r2}
  \begin{split}
    0 = a & + b \\
    = \frac{4301}{150}E^{-0.66} - \frac{85k}{3T}E^{0.165} - \frac{85}{3T}E^{0.34} & - \frac{5}{6T}E^{0.34} - \frac{5k^{0.2}}{6T}E^{0.165} \\
  \end{split}
\end{equation}
Gather like terms:
\begin{equation}\label{eq:dfde0r3}
  \begin{split}
  0 & = \frac{4301}{150}E^{-0.66} - \frac{85k}{3T}E^{0.165} - \frac{85}{3T}E^{0.34} - \frac{5}{6T}E^{0.34} - \frac{5k^{0.2}}{6T}E^{0.165} \\
    & = \frac{4301}{150}E^{-0.66} - \frac{170k + 5k^{0.2}}{6T}E^{0.165} - \frac{175}{6T}E^{0.34} \\
    \span \text{Let $C_1 = \frac{4301}{150}$, $C_2 = \frac{170k + 5k^{0.2}}{6T}$, and $C_3 = \frac{175}{6T}$} \\
    & = C_1E^{-0.66} - C_2E^{0.165} - C_3E^{0.34} \\
    \span \text{Let $E^\prime = E^{0.165}$. Note ${E^\prime}^2 = E^{0.34} \approx E^{0.33}$} \\
    & = C_1{E^\prime}^{-4} - C_2E^\prime - C_3{E^\prime}^2 \\
    & = C_1{E^\prime}^{-4} - C_2E^\prime - C_3{E^\prime}^{2*\frac{0.33}{0.34}} \\
    \span \text{Elide the constant $\frac{0.33}{0.34}$ as very close to $1$} \\
    & \approx C_1{E^\prime}^{-4} - C_2E^\prime - C_3{E^\prime}^2 \\
    \span \text{Assume $E^\prime \ne 0$, multiply by ${E^\prime}^4$} \\
    & = C_1 - C_2{E^\prime}^5 - C_3{E^\prime}^6 \\
    & = C_3{E^\prime}^6 + C_2{E^\prime}^5 - C_1 \\
  \end{split}
\end{equation}

There is no general solution for a sixth degree polynomial. We could use Newton-Raphson to iteratively solve for $E$ and $O$.

\hrulefill

TODO: Analyze an $f(E, M)$ form instead.

\begin{comment}
Substitute $k = \left(\frac{0.2}{0.34}\right)^{1.25}$:
\begin{equation}\label{eq:dfde0r4}
  \begin{split}
  0 & = \frac{4301}{150}E^{-0.66} - \frac{170\left(\frac{0.2}{0.34}\right)^{1.25} + 5\left(\frac{0.2}{0.34}\right)^{0.25}}{6T}E^{0.165} - \frac{175}{6T}E^{0.34} \\
    & = \frac{4301}{150}E^{-0.66} - \frac{170\left(\frac{0.2}{0.34}\right)^{1.25} + 5\left(\frac{0.2}{0.34}\right)^{0.25}}{6T}E^{0.165} - \frac{175}{6T}E^{0.34} \\
  \end{split}
\end{equation}
\end{comment}

%\newpage

\section{Analysis over E and M}

\subsection{Simplification}

Again, observe that B cannot have a positive effect on the objective function, and leaving the $T$ budget partially unallocated cannot improve the objective function.

Assign $B = 0$:

\[
f(E, O, M) = \left( E^{0.34} + O^{0.2} \right) \left( 1 + \frac{M}{1.2 \left( O + E + M \right)} \right)
\]

Substitute $O = T - E - M$:

\[
f(E, M) = \left( E^{0.34} + \left( T - E - M \right)^{0.2} \right) \left( 1 + \frac{M}{1.2 T} \right)
\]

\subsection{Optimization}

Take the partial derivatives with respect to $E$ and $M$.

\subsection{Find partial derivatives}

First, $\frac{\partial f}{\partial E}$:
\begin{equation}\label{eq:2dfde}
  \begin{aligned}
    f(E, M) & = \left( E^{0.34} + \left( T - E - M \right)^{0.2} \right) \left( 1 + \frac{M}{1.2 T} \right) \\
    \frac{\partial f}{\partial E} & = \frac{\partial}{\partial E}\left[\left( E^{0.34} + \left( T - E - M \right)^{0.2} \right) \left( 1 + \frac{M}{1.2 T} \right)\right] \\
    & = \frac{\partial}{\partial E}\left[
      \left( 1 + \frac{M}{1.2 T} \right) E^{0.34} + \left( 1 + \frac{M}{1.2 T} \right) \left( T - E - M \right)^{0.2}
    \right] \\
    & = 0.34\left( 1 + \frac{M}{1.2 T} \right)E^{-0.66} - \left( 1 + \frac{M}{1.2 T} \right)0.2\left( T - E - M \right)^{-0.8} \\
  \end{aligned}
\end{equation}

Then $\frac{\partial f}{\partial M}$. Using the product rule $\frac{\partial}{\partial M}(uv) = \frac{\partial u}{\partial M}v \cdot u + u\frac{\partial v}{\partial M}$. In:

\begin{equation}\label{eq:2fuv}
  f(E, M) = u \cdot v
\end{equation}
let
\begin{equation}\label{eq:2u}
  u = E^{0.34} + \left( T - E - M \right)^{0.2}
\end{equation}
and
\begin{equation}\label{eq:2v}
  v = 1 + \frac{M}{1.2 T}
\end{equation}

The derivatives are:
\begin{equation}\label{eq:2dudm}
  \frac{\partial u}{\partial M} = - 0.2\left( T - E - M \right)^{-0.8}
\end{equation}
and
\begin{equation}\label{eq:2dvdm}
  \frac{\partial v}{\partial M} = \frac{1}{1.2 T}
\end{equation}

From \eqref{eq:2fuv}, \eqref{eq:2u}, and \eqref{eq:2v}, we have:
\begin{equation}
  \begin{split}
    f(E, M) & = u \cdot v \\
    \frac{\partial f}{\partial M} & = \frac{\partial u}{\partial M}v + u\frac{\partial v}{\partial M}
  \end{split}
\end{equation}

Substitute \eqref{eq:2dudm}, \eqref{eq:2v}, \eqref{eq:2u}, and \eqref{eq:2dvdm} into the above equation:
\begin{equation}\label{eq:2dfdm}
  \begin{split}
    \frac{\partial f}{\partial M} & = - 0.2\left( T - E - M \right)^{-0.8}\left( 1 + \frac{M}{1.2 T} \right) + \left( E^{0.34} + \left( T - E - M \right)^{0.2} \right)\frac{1}{1.2 T} \\
    & = - 0.2\left( T - E - M \right)^{-0.8}\left( 1 + \frac{M}{1.2 T} \right) + \frac{E^{0.34} + \left( T - E - M \right)^{0.2}}{1.2 T} \\
  \end{split}
\end{equation}

\subsection{Find critical points}

Set the partial derivatives \eqref{eq:2dfde} and \eqref{eq:2dfdm} to zero:

\begin{equation}\label{eq:2dfde0}
  \begin{split}
    0 & = \frac{\partial f}{\partial E} \\
    & = 0.34\left( 1 + \frac{M}{1.2 T} \right)E^{-0.66} - \left( 1 + \frac{M}{1.2 T} \right)0.2\left( T - E - M \right)^{-0.8} \\
    & = 0.34\left( 1 + \frac{M}{1.2 T} \right)E^{-0.66} - 0.2\left( 1 + \frac{M}{1.2 T} \right)\left( T - E - M \right)^{-0.8} \\
    \span \text{$M$ and $T$ are positive, so $1 + \frac{M}{1.2T} \ne 0$} \\
    & = 0.34E^{-0.66} - 0.2\left( T - E - M \right)^{-0.8} \\
  \end{split}
\end{equation}
and
\begin{equation}\label{eq:2dfdm0}
  \begin{split}
    0 & = \frac{\partial f}{\partial M} \\
    & = - 0.2\left( T - E - M \right)^{-0.8}\left( 1 + \frac{M}{1.2 T} \right) + \frac{E^{0.34} + \left( T - E - M \right)^{0.2}}{1.2 T} \\
    & = \frac{E^{0.34} + \left( T - E - M \right)^{0.2}}{1.2 T} - 0.2\left( T - E - M \right)^{-0.8}\left( 1 + \frac{M}{1.2 T} \right) \\
    & = \frac{E^{0.34} + \left( T - E - M \right)^{0.2}}{1.2 T} - 0.2\left( 1 + \frac{M}{1.2 T} \right)\left( T - E - M \right)^{-0.8} \\
  \end{split}
\end{equation}

Rearrange \eqref{eq:2dfde0} and \eqref{eq:2dfdm0}, and dropping the common second summand:
\begin{equation}\label{eq:2dfde0r}
  \begin{split}
    0.34\left( 1 + \frac{M}{1.2 T} \right)E^{-0.66} & = \frac{E^{0.34} + \left( T - E - M \right)^{0.2}}{1.2 T} \\
    %0.34E^{-0.66} + \frac{0.34M}{1.2T}E^{-0.66} & = \\
    %1.2T \cdot 0.34E^{-0.66} + 0.34M E^{-0.66} & = E^{0.34} + \left( T - E - M \right)^{0.2} \\
    0.34\left(1.2T + M\right)E^{-0.66} & = E^{0.34} + \left( T - E - M \right)^{0.2} \\
    0.34\left(1.2T + M\right) & = E^{0.34}E^{0.66} + \left( T - E - M \right)^{0.2}E^{0.66} \\
    & = E + \left( T - E - M \right)^{0.2}E^{0.66} \\
  \end{split}
\end{equation}

This doesn't seem fruitful, and by symmetry of E and O, it seems unlikely that analysis over $O$ and $M$ would be any different.

\end{document}
